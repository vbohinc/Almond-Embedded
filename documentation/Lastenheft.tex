\documentclass[12pt,a4paper]{article}

\usepackage[german]{babel}
\usepackage[utf8]{inputenc}
\usepackage[pdftex]{graphicx}
\usepackage{fancyhdr,lastpage}
\usepackage{enumerate}
\usepackage{amsmath, amsthm, amssymb}
\usepackage[top=3cm, bottom=3cm, left=2cm, right=2cm]{geometry}
\usepackage{minitoc}
\pagestyle{fancy}

% Header
\lhead{Projekt Almond}
\chead{Lastenheft}
\rhead{Seite \thepage /\pageref{LastPage} }

\begin{document}

\begin{titlepage}
	\begin{center}
		\includegraphics[height=15cm]{./logo.pdf}\\
		{\LARGE \bf Lastenheft}\\[0.3cm]
		{\large und Projektübersicht}
	\end{center}
\end{titlepage}

\tableofcontents

\newpage

% Inhalt

\section{Übersicht}

Almond ({\bf A}utonomous {\bf L}ogging and {\bf M}anagement of {\bf N}etworked {\bf D}evices) ist ein System aus vernetzten Aktoren und Sensoren: Nuts ({\bf N}etworked {\bf Ut}ilities and {\bf S}ensors), die über einen zentralen Controller (Squirrel) gesteuert werden. Zusätzlich gibt es ein Backend für PCs, welches über eine Weboberfläche Zugriff auf Gerätedaten gibt und die Steuerung der Aktoren sowie Konfigurationsänderungen erlaubt.\\
Die einzelnen Geräte werden über ein dafür entwickeltes Protokoll via Bluetooth mit dem Controller verbunden. Der Controller selbst wird dann ebenfalls über Bluetooth mit dem auf einem Linux/Windows/Mac laufenden Backend verbunden. Dabei werden auch Logdaten (Verlauf der Sensor-Werte) mit übermittelt, die dann später auf dem Rechner ausgewertet werden können.

\section{Ziele}

\subsection{Endanwender}

Für Endanwender bietet Almond eine einfache Möglichkeit zur Überwachung und Automatisierung von Vorgängen mithilfe eines drahtlosen Netzes aus Geräten. \\
Dabei ist Almond im Gegensatz zu bestehenden Systemen

\begin{itemize}
	\item kostengünstig
	\item wartungsarm
	\item einfach zu installieren und zu benutzen
\end{itemize}

\subsection{Entwickler}

Für Entwickler bietet Almond ein Framework um drahtloses Steuern und Messen zu realisieren. \\
Dabei wird auf folgende Kriterien Wert gelegt

\begin{itemize}
	\item kostengünstig im vergleich zu bestehenden Systemen
	\item erweiterbar
	\item einfache Benutzung
	\item wartungsarm
	\item einfache Installation
\end{itemize}

\section{Teilnehmer und Aufgabenverteilung}

Den 9 Teilnehmern sind folgende Aufgabenbereiche im Projekt zugeteilt:\\

\begin{itemize}
	\item {\bf Salomon Sickert}: Teamleiter, Downlink-/Uplink Protokoll Design, Controller
	\item {\bf Pascal Schnurr}: Backend, Kommunikation mit Webinterface, Bluetooth Interface
	\item {\bf Matthias Schwab}: Backend, Webinterface
	\item {\bf Christian Rupprecht}: I/O Interface für Controller (SD Karte)
	\item {\bf Seán Labastille}: Controller, Up-/Downlink Protokolldesign
	\item {\bf Stefan Profanter}: Bluetooth Treiber für Aktoren/Sensoren, Backend
	\item {\bf Linus Lotz}: Hardwarelayout Aktoren/Sensoren, Controller
	\item {\bf Thomas Parsch}: Displaysteuerung des Controllers
	\item {\bf Maximilian Karl}: Hardwarelayout Aktoren/Sensoren, Controller
\end{itemize}
Die einzelnen Projektteile werden im Folgenden genauer erläutert.

\section{Controller}

\subsection{Aufbau}

Der Controller (``Squirrel'') basiert auf einem Atmel XMega 128a1 AVR Mikroprozessor. \\
Er dient als Zentrale für das System, welches autark vom Rechner arbeitet und auch als Schnittstelle zum Rechner. \\
Am Controller selbst befindet sich zusätzlich ein SD-Kartenleser für die Speicherung von Logdaten, ein Display mit Tasten für Statusüberprüfung.
Wie alle Client-Geräte (``Nuts'') besitzt der Controller ein Bluetooth Modul.

\subsection{Features}

Mit dem Downlink-Protokoll spricht der Controller die Client-Geräte an und empfängt Daten von diesen. Der Controller führt Buch über die ihm bekannten Geräte und speichert diese intern zwecks Identifizierung, Displayausgabe und Logverwaltung.\\
Mit dem Uplink-Protokoll kommuniziert der Controller mit dem Backend. Hiermit werden Daten (zum Beispiel zeitlicher Verlauf von Sensorwerten) zum Rechner geladen. \\

\subsection{Display}

Die grafische Benutzeroberfläche (GUI) des Controllers wird ebenfalls vom Atmel XMega 128a1 AVR Mikroprozessor berechnet und stellt somit eine der Grundfunktionen des Controllerboards dar.

\subsection{Speicherung (SD-Karte)}

Der Controller besitzt folgende Zugriffsmöglichkeiten auf SD-Karten:

\begin{enumerate}
	\item Log-Datei schreiben/lesen
	\item beliebige Textdatei schreiben/lesen
\end{enumerate}

So ist es möglich auf die Daten auch außerhalb des Controllers, z.B. vom PC aus zugreifen zu können.

\subsubsection{mögliche Erweiterungen}

Eine wünschenswerte Erweiterung wäre die Möglichkeit zeitgesteuert Aktoren auszulösen oder auf Zustandsänderungen bei einem Sensor zu reagieren.

\section{Aktor/Sensor Hardware}

\subsection{Aufbau}

Die Aktoren/Sensoren basieren auf dem Atmel AVR AtMega8535. Sie besitzen ein Bluetooth-Modul zur Kommunikation mit dem Controller. \\
Ein Designkriterium der Aktoren/Sensoren ist, dass sie einfach gehalten sind und wenig eigene Funktionalität bieten um kosten- und energiesparend zu sein.

\subsection{Anwendungsbeispiel: Wetterstation}

Die Wetterstation basiert auf der Aktor/Sensor Grundplatine mit zusätzlichen folgenden Sensoren:

\begin{itemize}
	\item Drucksensor (BMP085)
	\item Temperatursensor (BMP085)
	\item Windrichtungssensor (Eigenbau, optische Dekodierung)
	\item Windgeschwindigkeitssensor (Eigenbau, Dynamo)
	\item Lichtstärke (Fotowiderstand)
	\item Fechtigkeitssensor
\end{itemize}

Sie besitzt keine Aktoren und ist batteriebetrieben.

\subsubsection{mögliche Erweiterungen der Wetterstation}

\begin{itemize}
	\item weitere Sensoren wie Niederschlagssensor/Niederschlagsmenge
	\item erweiterte Unabhängigkeit der Wetterstation durch Windenergie/Sonnenenergie
\end{itemize}

\section{Protokolle}

Den Up-/Downlink Protokollen ist gemeinsam, dass Pakete fester Größe ausgetauscht werden.

\subsection{Uplink}
\label{subsec:uplink}

Dient der Kommunikation zwischen Backend und Controller, bei dem Logdaten und Konfiguration ausgetauscht werden.

\subsection{Downlink}
\label{subsec:downlink}

Dient der Kommunikation zwischen Aktor/Sensor Hardware und Controller.\\
Es werden Eigenschaften und Daten von Aktoren/Sensoren abgerufen (z.B. Sensordaten, Geräteinformationen sowie Aktorsteuerung). 

\subsection{Bluetooth}
Alle Geräte besitzen ein BTM-222 Bluetooth Modul zur Kommunikation untereinander sowie mit dem Computer (Backend). Zur Verbindung mit dem Computer wird ein bereits existierendes Bluetooth Modul am PC vorausgesetzt. \\
Sobald der Controller mit einem Aktor/Sensor oder dem PC verbunden ist, wird das Up- bzw. Downlink Protokoll genutzt.

\section{Backend}

\subsection{Aufbau}

Das Backend ist ein Softwarepaket für Linux, Windows oder Mac OS X basierte Computer. Mit dem Backend ist es möglich über eine webbasierte Oberfläche auf die Sensordaten der vernetzten Geräte zuzugreifen. \\
Der Vorteil der Weboberfläche ist, dass der Backend Server so von jedem Browser aus bedient werden kann (inklusive mobiler Endgeräte) und so eine bestmögliche Integration des Systems ermöglicht wird. \\
Als Webserver kommt der {\bf lighttpd} zum Einsatz. Zusätzlich wird eine lokale {\bf SQLite} Datenbank zur Speicherung von Einstellungen und Daten auf der Backend-Seite verwendet.\\
Die Software zur Kommunikation mit dem Controller wird in Java entwickelt und schreibt empfangene Daten direkt in die Datenbank.

\subsection{Backend Bluetooth Interface}
\label{subsec:BTBackend}

Das Backend Bluetooth Interface lädt regelmäßig neue Logdaten vom Controller und speichert diese in die MySQL Datenbank.\\
Außerdem werden geänderten Einstellungen oder Aktionen für Aktoren sofort wieder per Bluetooth in das Netz eingespielt.

\subsubsection{mögliche Erweiterungen}

\begin{enumerate}
	\item Benutzerverwaltung für das Webinterface
	\item Konfigurationseditor im Web
	\item evtl. verschlüsselte Kommunikation mit dem Controller
\end{enumerate}

\end{document}
