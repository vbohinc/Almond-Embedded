\documentclass[12pt,a4paper]{article}

\usepackage[german]{babel}
\usepackage[utf8]{inputenc}
\usepackage[pdftex]{graphicx}
\usepackage{fancyhdr,lastpage}
\usepackage{enumerate}
\usepackage{amsmath, amsthm, amssymb}
\usepackage[top=3cm, bottom=3cm, left=2cm, right=2cm]{geometry}
\usepackage{minitoc}
\pagestyle{fancy}

% Header
\lhead{Projekt Almond}
\chead{Aschlussbericht}
\rhead{Seite \thepage /\pageref{LastPage} }

\begin{document}

\begin{titlepage}
	\begin{center}
		\includegraphics[height=15cm]{./logo.pdf}\\
		{\LARGE \bf Abschlussbericht}\\[0.3cm]
	\end{center}
\end{titlepage}

\tableofcontents

\newpage

% Inhalt

\section{�bersicht}
Almond ({\bf A}utonomous {\bf L}ogging and {\bf M}anagement of {\bf N}etworked {\bf D}evices) ist ein System aus vernetzten Aktoren und Sensoren: Nuts ({\bf N}etworked {\bf Ut}ilities and {\bf S}ensors), die �ber einen zentralen Controller (Squirrel) gesteuert werden. Zus�tzlich gibt es ein Backend f�r PCs, welches �ber eine Weboberfl�che Zugriff auf Ger�tedaten gibt und die Steuerung der Aktoren sowie Konfigurations�nderungen erlaubt.\\
Die einzelnen Ger�te werden �ber ein daf�r entwickeltes Protokoll via Bluetooth mit dem Controller verbunden. Der Controller selbst wird dann ebenfalls �ber Bluetooth mit dem auf einem Linux/Windows/Mac laufenden Backend verbunden. Dabei werden auch Logdaten (Verlauf der Sensor-Werte) mit �bermittelt, die dann sp�ter auf dem Rechner ausgewertet werden k�nnen.

\section{Protokolle}
Um die Kommunikation zwischen den Nuts und dem Squirrel, sowie des Squirrels zum Backend zu gew�hrleisten sind zwei Protokolle entworfen worden.
	\subsection{Downlink}  
Das Downlinkprotokoll dient zur Kommunikation der Nuts mit dem Squirrel.
	\subsection{Uplink}
Das Uplinkprotokoll dient zur Kommunikation zwischen dem Squirrel und dem Backend.
	\subsection{UART}
	\subsection{SPI}

\section{Hardware}
	\subsection{Squirrel}
%realtimeclock erw�hnen und tasten
	\subsubsection{Bluetooth}
	\subsubsection{Display}
	\subsubsection{SD-Kartenleser}
	\subsubsection{}
	\subsection{Nut}


\section{Wetterstation}
	\subsection{Funktionen}
	\subsection{Hardware}
		\subsubsection Drucksensor (BMP085)
		\subsubsection Temperatursensor (BMP085)
		\subsubsection Windrichtungssensor (Eigenbau, optische Dekodierung)
		\subsubsection Windgeschwindigkeitssensor (Eigenbau, Dynamo)
		\subsubsection Lichtst�rke (Fotowiderstand)
		\subsubsection Fechtigkeitssensor

\section{Team}

\section{Entwicklungeschichte}
\subsection{Display}
Zun�chst wurde zum Entwickeln des Display-Treibers auf eine Test-Hardware zur�ckgegriffen, in der ebenfalls ein 13BB0-Display verbaut worden war.\\
Das urspr�nglich in einem DVB-T Receiver eingesetzte Board verf�gt jedoch nur nur �ber den schw�cheren ATMEGA 8515, welcher �ber weniger Speicher verf�gt.\\
Daher wurde zun�chst ein Programm entwickelt, das nur wenig Platz verbraucht und trotzdem die notwendige Flexibilit�t bietet, beliebige Inhalte anzuzeigen.\\
Um Buchstaben komfortabel f�r dieses Programm codieren zu k�nnen, wurde ein Hilfsprogramm in Java erstellt, welches eine Grafische Oberfl�che bietet zum Entwerfen der Buchstaben.\\
Im Laufe der Entwicklung stellte sich jedoch heraus, dass nicht nur eine Darstellung von Buchstaben an bestimmten fixen Pl�tzen auf dem Display ben�tigt wird, sondern eine noch flexiblere bitweise Ansteuerung.\\
Da die end�ltige Hardware des Squirrels �ber mehr Speicher verf�gt, war dies durch eine Neuschreiben weiter Teile des Display Treibers m�glich. Um das den weitgehen neuen Display Treiber mit verschiedenen Schriftarten versorgen zu k�nnen, wurde ein Perl-Skript geschrieben, welches die Standardfonts der GD-Libary konvertieren kann.\\



\subsection{Bluetooth}

\section{Quellenverzeichnis}


\end{document}